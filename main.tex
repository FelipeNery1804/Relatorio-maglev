%===============================================================================
% $Id: ifacconf.tex 19 2011-10-27 09:32:13Z jpuente $  
% Template for IFAC meeting papers
% Copyright (c) 2007-2008 International Federation of Automatic Control
%===============================================================================
\documentclass{ifacconf}

\usepackage{graphicx}      % include this line if your document contains figures
\usepackage{natbib}        % required for bibliography
\usepackage{amsmath}
%===============================================================================
\begin{document}
\begin{frontmatter}

\title{Grupo 3 - Módulo 2 - Maglev} 
% Title, preferably not more than 10 words.

%\thanks[footnoteinfo]{Sponsor and financial support acknowledgment
%goes here. Paper titles should be written in uppercase and lowercase
%letters, not all uppercase.}

\author[First]{Felipe Nery Barcelos Araújo (2020021190)} 
\author[First]{Gustavo Vieira Barbosa (2020021352)} 
\author[First]{Matheus Marques Gonçalves de Paula (2020068995)}

\address[First]{
  Engenharia de Controle e Automação,\\ Universidade Federal de Minas Gerais, MG \\
   (e-mails: felipenery@ufmg.br, gustavovbarbosa@ufmg.br, mmgp@ufmg.br)
}

%Escrever um resumo do documento aqui, não pode ultrapassar 250 palavras
\begin{abstract}               
Relatório do controle de posição de uma massa levitada por um eletroima
\end{abstract}

%Escrever até 5 palavras chave do relatório aqui
\begin{keyword}
maglev, controle, posição, levitação, eletroima
\end{keyword}

\end{frontmatter}

%===============================================================================

\section{Introdução}

A levitação magnética, é uma tecnologia em que utiliza forças magnéticas para suspender e controlar a posição de
objetos metálicos, como uma esfera de aço, sem a necessidade de contato físico com superfícies. Um exemplo notável dessa aplicação é a levitação de uma esfera
de aço por meio da força magnética gerada por um eletroímã, que será abordado nesse relatório por meio da planta Maglev da \textit{Feedback}. Tal controle tem importância significativa 
tanto na sociedade como na indústria, oferecendo uma ampla gama de benefícios, desde sistemas de transporte de alta velocidade até aplicações em pesquisa e desenvolvimento.

Ao longo desse relatório será visto um estudo focado no controle de posição de uma esfera de aço por meio de forças magnéticas 
gerada por um eletroímã, como mostra a figura \ref{fig:planta_padrao}, um problema
clássico de controle de sistemas magnéticos. 

\begin{figure}[!htb]
  \begin{center}
  \includegraphics[width=6.4cm]{figures/planta_padrao.png}    % The printed column width is 8.4 cm.
  \caption{Figura da planta real estudada, levitador magnético (Maglev). Fonte: Autoral} 
  \label{fig:planta_padrao}
  \end{center}
\end{figure}

Com isso, nas seções subsequentes, exploraremos em detalhes a modelagem matemática do sistema
maglev, bem como o fundamento do controle utilizado e demonstrações práticas.

\section{Descrição da planta e especificação do desempenho desejado}

O controle a ser realizado visa, inicialmente, controlar a posição de uma esfera de aço bem como tornar o sistema estável (deixando-a parada), erro nulo para entrada de degrau, 
%CONFIRMAR ISSO DO TEMPO DE ACOMADAÇÃO E SOBRESSINAL
tempo de acomodação menor do que 4 segundos e sobressinal até 20\%.
A planta a ser estudada envolve além da esfera de aço, um eletroímã, o qual irá gerar um campo magnético que
consequentemente irá impor uma força eletromagnética sobre a esfera fazendo-a levitar com um controle preciso de posição.


\section{Modelagem matemática e validação do modelo}



\subsection{Modelagem matemática}

\subsection{Modelagem do Sensor}

A variável quantificada pelo sensor corresponde à distância,
sendo que sua resposta resultante é expressa em termos de tensão.
Neste contexto, surge a necessidade de um modelo que estabeleça a
relação entre a resposta em tensão gerada pelo sensor e a distância
efetivamente medida por este último. Devido à proporcionalidade
das variáveis, a relação entre elas pode ser expressa pela seguinte
função:
\begin{equation}
    y = ax + b
\end{equation}
onde:
\begin{itemize}
    \item y - resposta do sensor em volts [V]
    \item x - distância medida em metros [m]
    \item a - constante de ganho do sensor em volts por metro [V/m]
    \item b - constante de offset do sensor em volts [V]
\end{itemize}

\subsection{Modelagem da Planta}

Inicialmente, procedemos com uma análise das forças atuantes
no sistema em questão. A força exercida pela bobina é formulada
mediante a seguinte expressão matemática:
\begin{equation}
    \vec{F_b} = k_1 \frac{i^2}{x^2}
\end{equation}

\noindent onde $i$ representa a intensidade da corrente elétrica, $x$ denota adistância entre o objeto e a bobina, e $k_1$ representa um coeficiente intrínseco ao circuito elétrico. Paralelamente, a força gravitacional que age sobre um corpo é definida como:

\begin{equation}
    \vec{F_g} = mg
\end{equation}

\noindent onde $i$ corresponde à massa do corpo em questão, e $g$ representa a aceleração devida à gravidade. Em concordância comos princípios da segunda lei de Newton, podemos estabelecer o
seguinte resultado:

\begin{equation}
    m\ddot{x} = mg - k_1 \frac{i^2}{x^2}
\end{equation}

Contudo, considerando que o sinal de entrada se encontra na forma de tensão, torna-se mais propício realizar a conversão da corrente para a mesma unidade de medida. Contudo, o Maglev tem uma malha de controle de corrente interna, possibilitando a seguinte relação funcional:

\begin{equation}
    i = k_2u
\end{equation}

\noindent onde $k_2$ representa um coeficiente constante e $u$ corresponde ao sinal de entrada em tensão. Substituindo a equação (5) em (4):

\begin{equation}
    m\ddot{x} = mg - k_1 \frac{(k_2u)^2}{x^2}
\end{equation}

Tendo as equações não lineares que representam o sistema, pode-se
construir a representação da dinâmica da planta por:

\begin{equation}
    \dot{x}_1 = x_2
\end{equation}
\begin{equation}
    \ddot{x}_2 = g - k_1 \frac{(k_2u)^2}{mx^2}
\end{equation}

A equação (8) pode ser simplificada para:

\begin{equation}
    f(x,u) = \ddot{x}_2 = g - \frac{k}{m} \frac{u^2}{x^2}
\end{equation}

\noindent facilitando a linearização.

\subsection{Linearização}

A teoria de controle linear parte do princípio de que a planta
possui comportamento linear. Mas nem sempre isto é
verdade. Contudo, para toda função bem comportada em torno de
um ponto fixo, para variações pequenas, curva pode ser aproximada por uma reta que passa pelo ponto fixo. Novas equações podem ser obtidas a partir de (7) e (9) de modo a se obter a dinâmica para pequenas variações. Por este motivo as seguintes variáveis são definidas:

\begin{equation}
    \delta x = x - x_{eq}
\end{equation}
\begin{equation}
    \delta u = u - u_{eq}
\end{equation}
\begin{equation}
    \delta y = y - y_{eq}
\end{equation}

\noindent onde $x$, $u$ e $y$ são as variáveis originais, $x_{eq}$, $u_{eq}$ e $y_{eq}$ são valores que a planta assume quando está em uma condição de equilíbrio e $\delta x$, $\delta u$ e $\delta y$ as variações em torno do ponto de equilíbrio. Vale ressaltar que o ponto de equilíbrio escolhido deve ser o mais próximo possível das especificações de desempenho desejado, ou seja, os valores que fazem a esfera flutuar a 9 mm, pois fora do ponto de equilíbrio os comportamentos não lineares são mais evidentes. Substituindo o novo sistema de variáveis nas equações de estados a seguintes equações são obtidas:

\begin{equation}
    \dot{x}_1 = x_2 = 0
\end{equation}
\begin{equation}
    \ddot{x}_2 = g - k \frac{u_o^2}{x_o^2} = 0 \rightarrow u_o = \sqrt{\frac{mg}{k}}x_o
\end{equation}
Os valores de $m$ e $k$ não são definidos separadamente, mas sim através da seguinte relação:
\begin{equation}
    \frac{k}{m} = 1,2415 \cdot 10^{-3}
\end{equation}
\noindent enquanto a aceleração da gravidade é tomada por $g = 9,81 \; m/s^2$. Sendo assim, a corrente de entrada para uma saída de equilíbrio $x_o = 9\;mm$ é:
\begin{equation}
    u_o = \sqrt{\frac{9,81\;m/s^2}{1,2415 \cdot 10^{-3}}} \cdot 0,009\;m
\end{equation}
\begin{equation}
    u_o = 0,8\;A
\end{equation}

O modelo linearizado da planta, portanto, é dado por:
\begin{equation}
    \ddot{x} = -[K_i u + a_x x]
\end{equation}
onde $K_i$ e $a_x x$ são coeficientes dados por:
\begin{equation}
    K_i = [\frac{\partial}{\partial u}f(x,u)]_{x = x_o, u = u_o}
\end{equation}
\begin{equation}
    a_x = [\frac{\partial}{\partial x}f(x,u)]_{x = x_o, u = u_o}
\end{equation}
Tomando as derivadas parciais e aplicando o ponto de operação, tem-se:

\begin{equation}
    K_i = \frac{-2ku_o}{mx_o^2} = \frac{-2\cdot1,2415 \cdot 10^{-3}\cdot 0,8\;A}{(0,009\;m)^2} = 
\end{equation}
\begin{equation}
    a_x = \frac{2ku_o^2}{mx_o^3} = \frac{2\cdot1,2415 \cdot 10^{-3}\cdot (0,8\;A)^2}{(0,009\;m)^3} = 
\end{equation}

\section{Projeto do controlador}

A principio, buscamos realizar o controlador proporcional, integrativo e derivativo (PID), 
por garantir erro nulo para entrada em degrau e por ser amplamente difundido nas industrias
e sistemas de controle em geral. A seguir será explicitado as tentativas para obter esse controlador.

\subsection{Primeira tentativa} %Falha obtida
Para obtenção dos parâmetros foi plotado o lugar das raizes da planta em malha fechada, figuras \ref{fig:lugar_raizes_mf} e \ref{fig:lugar_raizes_mf_aproximada}.
Com o lugar da raizes traçados, foi utilizado o \textit{sisotool} para realizar a escolha da localização dos polos e ganho do controlador PID, com isso realizamos
o cancelamento de polos mais instaveis, os polos localizados mais a direita do circulo unitário, e escolhemos um ganho arbritário para sair da instabilidade e obter
uma resposta satisfatória. Por fim, implementamos o controle na planta e visualizamos que a resposta obtida na planta não condizia com o esperado visto no \textit{MatLab},
independente da referência inserida, o carrinho não se mexia nada, consequentemente, acreditamos que o ganho do Kp estava baixo e fomos aumentando de forma empirica, 
após diversos aumentos no Kp, o carrinho começou a se mexer. Sendo assim, a planta modelada era equivalente com o sistema em malha aberta, mas em malha fechada o comportamento
era diferente. De forma a elucidar o ocorrido, assumimos de que a falha ocorreu devido aos coeficientes de atrito C1 e C2, portanto, aumentamos esses parâmetros, empiricamente, até
obter uma resposta satisfatória para iniciar novamente o processo de controle mas com uma nova modelagem da planta. 

\begin{figure}[!htb]
  \begin{center}
  \includegraphics[width=8.4cm]{figures/lugar_raizes_mf.png}    % The printed column width is 8.4 cm.
  \caption{Lugar da raizes da malha fechada da planta. Fonte: autoral, produzida com \textit{matlab}.} 
  \label{fig:lugar_raizes_mf}
  \end{center}
\end{figure}

\begin{figure}[!htb]
  \begin{center}
  \includegraphics[width=8.4cm]{figures/lugar_raizes_mf_aproximada.png}    % The printed column width is 8.4 cm.
  \caption{Lugar da raizes da malha fechada da planta aproximada na extremidade do circulo unitário. Fonte: autoral, produzida com \textit{matlab}.} 
  \label{fig:lugar_raizes_mf_aproximada}
  \end{center}
\end{figure}

\subsection{Segunda tentiva} %Controle corrigido
Sequencialmente após a correção dos parâmetros C1 e C2, de forma impirica, recalculamos o modelo de espaço de estados utilizando a equação X.

Posteriormente foi realizado um novo estudo sobre o lugar das raizes em malha fechada, figura \ref{fig:lugar_raizes_mf_atualizada}, em que
aproximando para próximo da extremidade do circulo unitário temos a figura \ref{fig:lugar_raizes_mf_atualizada_zoom}. 
Em que é possível perceber que a caracteristica da resposta do modelo é fortemente marcada por dois pares de polos conjugados
próximos a extremidade do circulo unitário levando para a instabilidade do sistema, como mostra a resposta ao degrau do sistema, figura \ref{fig:resposta_Degrau_mf_atualizada}.

\begin{figure}[!htb]
  \begin{center}
  \includegraphics[width=8.4cm]{figures/lugar_raizes_mf_corrigida.png}    % The printed column width is 8.4 cm.
  \caption{Lugar da raizes da malha fechada da planta atualizada. Fonte: autoral, produzida com \textit{matlab}.} 
  \label{fig:lugar_raizes_mf_atualizada}
  \end{center}
\end{figure}

\begin{figure}[!htb]
  \begin{center}
  \includegraphics[width=8.4cm]{figures/lugar_raizes_mf_corrigida_aproximada.png}    % The printed column width is 8.4 cm.
  \caption{Lugar da raizes da malha fechada da planta atualizada aproximada na extremidade do circulo unitário. Fonte: autoral, produzida com \textit{matlab}.} 
  \label{fig:lugar_raizes_mf_atualizada_zoom}
  \end{center}
\end{figure}

\begin{figure}[!htb]
  \begin{center}
  \includegraphics[width=8.4cm]{figures/resposta_degrau_mf_atualizado.png}    % The printed column width is 8.4 cm.
  \caption{Resposta ao degrau da planta atualizada, sem controlador, indo para a instabilidade. Fonte: autoral, produzida com \textit{matlab}.} 
  \label{fig:resposta_Degrau_mf_atualizada}
  \end{center}
\end{figure}

Com isso, foi inserido o integrador e um par de zeros conjugados, controlador PID, próximo ao par de polos conjugados mais a direita do circulo unitário com 
a finalidade de atraí-los para dentro do circulo unitário, estabilizando o sistema e melhorando a margem de ganho. Nesse processo visualizamos
que o integrador colocado fazia com que o lugar das raizes do segundo par de polos conjugados da planta direcionasse para a instabilidade, portanto,
procuramos alocar o par de zeros conjugados de forma a garantir a não instabilidade do primeiro par de polos e de aumentar a margem de ganho em relação
ao segundo par de polos, para tal, o par de zeros deveria se localizar entre ambos os polos, mas mais próximo do primeiro. O lugar das raizes do controlador
é explicitado pelas figuras \ref{fig:lugar_raizes_mf_atualizada_controlador} e \ref{fig:lugar_raizes_mf_atualizada_controlador_zoom}. 

\begin{figure}[!htb]
  \begin{center}
  \includegraphics[width=8.4cm]{figures/lugar_raizes_controlador_finalpng.png}    % The printed column width is 8.4 cm.
  \caption{Lugar da raizes da malha fechada da planta atualizada com controlador PID. Fonte: autoral, produzida com \textit{matlab}.} 
  \label{fig:lugar_raizes_mf_atualizada_controlador}
  \end{center}
\end{figure}


\begin{figure}[!htb]
  \begin{center}
  \includegraphics[width=8.4cm]{figures/lugar_raizes_controlador_final_zoom.png}    % The printed column width is 8.4 cm.
  \caption{Lugar da raizes da malha fechada da planta atualizada com controlador PID aproximada na extremidade do circulo unitário. Fonte: autoral, produzida com \textit{matlab}.} 
  \label{fig:lugar_raizes_mf_atualizada_controlador_zoom}
  \end{center}
\end{figure}

Posteriormente, foi ajustado o ganho do controlador visualizando a resposta ao degrau do sistema com controlador, tendo como foco que o tempo de acomodação
fosse menor do que 4 segundos e principalmente que não houvesse sobressinal, pois como a planta possui uma margem de apenas 3 centimetros, sendo que nos 3 centimetros
possui uma chave fim de curso em que corta a alimentação do motor, o sobressinal atrapalha muito o controle, a reposta ao degrau obtida é representada pela curva 'Modelado' na figura \ref{fig:comparacao_mf_controle_step}.
Por fim, a equação do controlador PID obtido está explicitada na equação \ref{eq:controlador}.

\begin{equation}
  C = \frac{0,27951(z^2 - 1,994z + 0,9948)}{(z-1)}
  \label{eq:controlador}
\end{equation}

% Retirei essa figura para poder diminuir no espaço, uma vez q essa curva está dentro de outro gráfico
%\begin{figure}[!htb]
%  \begin{center}
%  \includegraphics[width=8.4cm]{figures/resposta_degrau_controlador_final.png}    % The printed column width is 8.4 cm.
%  \caption{Resposta ao degrau da malha fechada da planta atualizada com controlador PID. Fonte: autoral, produzida com \textit{matlab}.} 
%  \label{fig:resposta_degrau_mf_atualizado_controlador}
%  \end{center}
%\end{figure}

No final, o controlador foi implementado e aplicado um degrau unitário na planta com o objetivo de comparar o resultado real com o esperado
pela modelagem, essa comparação é representada pela figura \ref{fig:comparacao_mf_controle_step}.

\begin{figure}[!htb]
  \begin{center}
  \includegraphics[width=8.4cm]{figures/comparacao_mf_controle_step.png}    % The printed column width is 8.4 cm.
  \caption{Comparação da resposta ao degrau em MF da planta em relação ao modelo, com controlador PID. Fonte: autoral, produzida com \textit{matlab}.} 
  \label{fig:comparacao_mf_controle_step}
  \end{center}
\end{figure}

Nessa comparação é possível perceber uma dinâmica aproximada entre os modelos, todavia, há algumas divergências nas amplitudes do movimento do período transitório
e demora para estabilização no regime permanente.  

\section{Resultados experimentais em malha fechada}

Essa seção testa o controle obtido em diversas situações e entradas diferentes na planta.

% Coloquei esse gráfico dentro da parte de controle comparando com o modelo
%\begin{figure}[!htb]
%  \begin{center}
%  \includegraphics[width=8.4cm]{figures/resultado_teste4.png}    % The printed column width is 8.4 cm.
%  \caption{Resposta ao degrau unitário da planta, carro 1 e 2 com 4 pesos. Fonte: autoral, produzido com \textit{matlab} por meio dos dados coletados na planta.} 
%  \label{fig:teste_step1_c1_4p_c2_4p}
%  \end{center}
%\end{figure}

\begin{figure}[!htb]
  \begin{center}
  \includegraphics[width=8.4cm]{figures/resultado_teste1.png}    % The printed column width is 8.4 cm.
  \caption{Resposta ao degrau unitário da planta, carro 1 com 1 peso e carro 2 com 4 pesos. Fonte: autoral, produzido com \textit{matlab} por meio dos dados coletados na planta.} 
  \label{fig:teste_step1_c1_1p_c2_4p}
  \end{center}
\end{figure}

\begin{figure}[!htb]
  \begin{center}
  \includegraphics[width=8.4cm]{figures/resultado_teste2.png}    % The printed column width is 8.4 cm.
  \caption{Resposta ao degrau unitário da planta, carro 1 e 2 com 1 peso. Fonte: autoral, produzido com \textit{matlab} por meio dos dados coletados na planta.} 
  \label{fig:teste_step1_c1_1p_c2_1p}
  \end{center}
\end{figure}

\begin{figure}[!htb]
  \begin{center}
  \includegraphics[width=8.4cm]{figures/resultado_teste3.png}    % The printed column width is 8.4 cm.
  \caption{Resposta ao degrau unitário da planta, carro 1 e 2 com nenhum peso. Fonte: autoral, produzido com \textit{matlab} por meio dos dados coletados na planta.} 
  \label{fig:teste_step1_c1_0p_c2_0p}
  \end{center}
\end{figure}

\begin{figure}[!htb]
  \begin{center}
  \includegraphics[width=8.4cm]{figures/resultado_teste5.png}    % The printed column width is 8.4 cm.
  \caption{Resposta a rampa com inclinação de 0,5cm/s e saturação em 2cm, carro 1 e 2 com 4 pesos. Fonte: autoral, produzido com \textit{matlab} por meio dos dados coletados na planta.} 
  \label{fig:teste_ramp1_c1_4p_c2_4p}
  \end{center}
\end{figure}

\begin{figure}[!htb]
  \begin{center}
  \includegraphics[width=8.4cm]{figures/resultado_teste6.png}    % The printed column width is 8.4 cm.
  \caption{Resposta a senoide com amplitude de 2cm e frequencia de 1 Hz, carro 1 e 2 com 4 pesos. Fonte: autoral, produzido com \textit{matlab} por meio dos dados coletados na planta.} 
  \label{fig:teste_sin1_c1_4p_c2_4p}
  \end{center}
\end{figure}

Os testes exemplificados acima, mostram que o controle foi eficaz em sua maioria, pois foi capaz de atingir a referência para entradas de degrau unitário e conseguiu também ratrear outros tipo de entrada,
isso mostra que apesar de simples, o controle PID é capaz de atingir requisitos específicos em uma planta e também consegue estabilizar os sitemas em sua maioria por possibilitar o posicionamento de dois zeros
e um polo dentro da planta, sendo uma opção fácil e de eficaz impementação.    

\section{Conclusão}

Por fim, vale ressaltar que o projeto envolveu o estudo da modelagem e controle de uma planta linear e serviu de base para o estudo prático das teorias de controle estudadas ao longo de disciplinas de semestres anteriores. A modelagem envolveu o uso de uma abordagem fenomenológica para a decomposição das forças do sistema e a partir disso os parâmetros da planta foram encontrados com a realização de testes envolvendo a troca de molas, pesos e a fixação dos carrinhos. Depois de modelar o sistema, foi desenvolvido um controlador com base no método do lugar das raízes com o \textit{software} \textit{Matlab} e com isso os parâmetros da resposta foram aprimorados para maior confiabilidade do modelo. Com isso, os parâmetros do PID foram implementados de forma discreta no controlador do sistema massa mola e diversos testes com entradas diferentes foram feitos.  

\section{Referências}

\begin{itemize}
  \item Notas de aula
  \item Manual da planta
\end{itemize}

\end{document}