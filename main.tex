%%%%%%%%%%%%%%%%%%%%%%%%%%%%%%%%%%%%%%%%%%%%%%%%%%%%%
% Exemplo 03-01: fontes: familia, serie, forma, e tamanho
% ultima atualizacao: 04/02/2018 por Sadao Massago
% http:/www.dm.ufscar.br/~sadao
%----------------------------------------------------
\documentclass[12pt,a4paper]{article}

% T1 não é aceito no PCTeX 4.0 ou anterior.
% Neste caso, comente
\usepackage[T1]{fontenc} % codificação da fonte em 8-bits
\usepackage[utf8]{inputenc} % acentuação direta
\usepackage[brazil]{babel} % em portugues brasileiro

% regra de hifenização das palavras nao acentuadas:
% nao requer \usepackage[T1]{fontenc}
\hyphenation{li-vro tes-te cha-ve bi-blio-te-ca}
% regra de hifenização das palavras acentuadas:
% requer \usepackage[T1]{fontenc}
\hyphenation{co-men-tá-rio re-fe-rên-cia}

\sloppy % preferencia a underflow (do que overflow)

\begin{document} % inicio do documento

No \LaTeX, as fontes é especificado pela família ou categoria (como ele é desenhado), serie ou peso (espessura da letra -- negrito ou não), 
e a forma (formato da letra como itálico e não itálico).

\

{\bfseries comandos de fontes}
O \LaTeX{} dispõe de dois tipos de comandos para especificação de fontes.
Um deles é a especificação num trecho que é da forma
\verb+\text??{texto}+ onde \verb+??+ é a seqüência de duas letras
e \verb+texto+ é o trecho do texto com tal especificação da fonte.

Por exemplo, \verb+\textbf{negrito}+ produz \textbf{negrito}.

Outro comando é a especificação da seleção de fontes que pode afetar 
o documento todo (ou todo ambiente, se for colocado dentro do ambiente).
Por exemplo, \verb+\bfseries+ troca para negrito, 
até que apareça algum comando desfazendo esta ação
(ou \verb+\end{}+).

Ex.: \bfseries o texto ficará em negrito até que apareça comando como
\verb+\normalfont+. 
\normalfont Agora, a fonte retornou ao normal.

O comando de seleção de fontes podem ser usados para mudança apenas de 
um trecho. Para tanto, basta colocar entre chaves, como em

Ex.: {\bfseries texto em negrito somente dentro da chave}. 
Aqui, a fonte voltou ao normal.

Um cuidado especial no caso do uso acima é que o comando de seleção 
de fontes não leva em conta o ajuste de espaços necessário para que 
a mudança fique suave. Por exemplo, última letra da fonte itálico 
seguido de fonte reto pode ficar grudado demais. No exemplo acima, 
pode ser usado por terminar a frase com ponto final, 
no ato da troca de fontes.

Quando precisa especificar um trecho grande de paragrafos, 
é recomendado que use o comando de mudança de fontes 
na forma de ambientes, isto é, colocando em
\verb+\begin{}+ e \verb+\end{}+ como a seguir, 
recomendado quando for especificação de um parágrafo por deixar 
claro onde começa e onde termina a especificação.

Ex.:
\begin{bfseries}
    Este trecho ficará todo em negrito.
\end{bfseries}

{\bfseries family (familia):}

O comando \verb|\rmfamily| seleciona a fonte com serifa 
(enfeite nas pontas, padrão para maioria dos casos) 
indicadas para corpo do documento. 
comando para trecho do texto para esta família é \verb+\textrm{texto}+.

Ex.: \textrm{fonte Romano} ou {\rmfamily Fonte com serifa}.

O comando \verb|\sffamily| seleciona a fonte sem serifa 
(sem enfeite nas pontas) indicadas para títulos, 
transparências, cartazes, etc. 
No caso da classe \texttt{amsart} e \texttt{amsbook}, usa esta fonte 
nos títulos, mas \texttt{article}, \texttt{book}, \texttt{report}, etc 
não usa esta família.
O comando para trecho de texto é \verb+\textsf+.

Ex.: \textsf{Sans Serif} ou {\sffamily Sem Serifa}.

O comando \verb|\ttfamily| seleciona a fonte monoespaçado (sem serifa, 
com espaçamento constante) usado para escrever código fonte de programa,
nome de arquivos, endereço de internet, etc que deve ser lido no pé da letra.
O comando \verb|\verb| usa esta fonte.
O comando para trecho de texto é \verb+\texttt+.

Ex.: \texttt{Typewriter} ou {\ttfamily fonte de máquina de escrever}.

\

{\bfseries series (peso):}

Para destacar a parte do documento, tal como título, 
costuma usar a fonte em negrito.

O comando \verb|\mdseries| seleciona a espessura média 
(normal - padrão para maioria dos casos).
O comando para trecho de texto é \verb+\textmd+.


Ex.: \textmd{Peso Normal} ou {\mdseries espessura normal}.


O comando \verb|\bfseries| seleciona o negrito (mais grosso).
O comando para trecho de texto é \verb+\textbf+.


Ex.: \textbf{Negrito} ou 
\begin{bfseries}
negrito
\end{bfseries}
ou {\bfseries espessura grossa}.

Ex.: \begin{center}
\bfseries Centralizado e em negrito
\end{center}

\

{\bfseries shape (forma):}

O comando \verb|\upshape| seleciona a letra reta (padrão para maioria dos casos).
O comando para trecho de texto é \verb+\textup+.

Ex.: \textup{Reto} ou {\upshape forma reta}.

O comando \verb|\itshape| seleciona a letra itálico (inclinado e arredondado).
O comando para trecho de texto é \verb+\textit+.

Ex.: \textit{Itálico} ou {\itshape fonte arredondado}.

Ainda tem mais duas formas que requer cuidado no uso que são:

\verb|\slshape| que seleciona a letra inclinada, mas não arredondada
(pode ser usado no cabeçalho, etc, mas evite de usar no meio do texto)
O comando para trecho de texto é \verb+\textsl+.

Ex.: \textsl{Inclinado} ou {\slshape não é itálico}.

O comando \verb|\scshape| seleciona a letra maiuscula de tamanho variado.
O comando para trecho de texto é \verb+\textsc+.

Ex.: \textsc{Small Caps} ou {\scshape Não está em maiúsculo}.

{\bfseries fonte normal e texto em ênfase:}

Quando quer voltar a configuração padrão do documento, 
comando \verb|\normalfont| que restaura a configuração do \verb|family|, \verb|series| e \verb|shape|. 
O comando para trecho de texto é \verb+\textnormal+.

Ex.:
{\bfseries \itshape Este trecho é negrito itálico, 
mas \textnormal{aqui voltou ao normal} e 
agora continua com negrito itálico}

ou 

{\bfseries \itshape Este trecho é negrito itálico, mas pode voltar 
ao normal.
{\normalfont Aqui voltou ao normal}, 
mas continua com negrito itálico}
 

Para dar ênfase uma parte do texto, mas não para destacar, 
costuma alterar entre fonte reto e itálico.
O comando para seleção do enfatizado é \verb|\em| e para trecho do texto 
é \verb+\emph+ (não é \verb+textem+)

O uso de \verb+\textit{}+ (ou \verb|\itshape|) é indicado somente 
para casos em que ele deve ficar em itálico, independente do contexto, 
tal como enunciado do teorema, mas para enfatizar 
(como termo que está sendo definido), deverá usar o \verb+emph+ 
(ou \verb|\em|).

Ex.: 
\begin{em}
  As veses, é preciso criar \emph{ênfase} dentro do texto já em ênfase. 
\end{em}.

\

{\bfseries tamanho:}

Comando para definir tamanho, de menor para maior são:

\verb|\tiny| {\tiny tiny}

\verb|\scriptsize| {\scriptsize scriptsize}

\verb|\footnotesize| {\footnotesize footnotesize}

\verb|\normalsize| {\normalsize normalsize}

\verb|\large| {\large large}

\verb|\Large| {\Large Large}

\verb|\LARGE| {\LARGE LARGE}

\verb|\huge| {\huge huge}

\verb|\Huge| {\Huge Huge}

Estes comandos devem ser usados, protegendo com chaves como em

Ex.: {\Large fonte grande}

Quando for um paragrafo inteiro, é recomendável que use o
\verb+\begin+ e \verb+\end+ como em

\begin{small}
    Está com a letra pequena.
\end{small}

Note que, quando a letra fica pequeno demais devido ao tamanho base 
do documento, \texttt{tiny} e \texttt{scriptsize} pode 
ficar do mesmo tamanho.

\

{\bfseries Cuidado quando especifica a fonte no meio do texto:}

Para usar fonte diferente no meio do texto, requer cuidados especiais. 
Por exemplo, uma fonte reto seguido de inclinado requer espaço extra entre 
eles para evitar que fiquem muito próximos.
O comando de seleção de fontes (como o \verb+{\itshape texto}+)
não verifica este tipo de comportamento.

Para resolver este tipo de problemas, existem os comandos \verb|\text??{}| 
onde \verb|??| é a primeiras duas letras usadas no comando da seleção das 
fontes: familia, peso ou forma.
Assim, no meio do texto, sempre opte pelo grupo \verb+\text??+ em vez do comando de seleção de fontes.

Os seguintes comandos especificam o texto na fonte indicada que são:
\verb|textrm{texto}|, \verb|\textsf{texto}|, \verb|\texttt{texto}|, \verb|\textmd{texto}|,
\verb|\textbf{texto}|, \verb|\textup{texto}|, \verb|\textit{texto}|, \verb|\textsl{texto}|,
\verb|\textsc{texto}|, \verb|\textnormal{texto}|, \verb|\emph{texto}|.

O \verb|textnormal| (corresponde a \verb|\normalfont|) e
 \verb|\emph{}| (corresponde a \verb|\em|) 
são as excessões, mas todos outros usam como terminação, 
duas letras inicias usadas na especificação correspondente.

Portanto, qualquer especificação de um trecho de texto dentro do paragrafo, 
use o comando de especificação do trecho, e não a seleção da fonte.

\

{\bfseries Bloco de parâgrafos:}

Para especificar um bloco de texto consideravelmente grande, 
usa-se a forma de ambiente correspondente 
as formas declarativas. 
Por exemplo, o texto compreendidas entre 
\verb|\begin{itshape}| e \verb|\end{itshape}| ficam em itálico.
 Todos comandos de seleção de fontes (que não sejam \verb+\text??+ ou \verb+emph+), assim como especificações do tamanho aceitam a 
 forma de ambientes.

\

{\bfseries Combinação:}

Primeiro, observe que as especificações das fontes podem ser combinadas, 
desde que tais combinações existam.
Ex.: 
\begin{itshape} 
  Está em itálico \textbf{negrito} ou \texttt{typewriter} 
\end{itshape}

\

Quando a especificação da fonte for efetuado dentro de um bloco 
(delimitado pelos chaves) ou dentro do ambiente 
(delimitado por algum \verb|\begin{ambiente}| e \verb|\end{ambiente}|, 
ele não afeta o que está fora dele. 
O caso do ambiente funciona no caso mais geral que a seleção de fontes
e tamanho de fontes. 

\textbf{Nota:} O ambiente (com \verb+\begin{}+ e \verb+\end{}+
protege o resto do documento para que possa efetuar quaisquer 
alteração de configuração, incluindo alinhamento de textos 
(mesmo dentro do ambiente, as configurações globais tal como 
configuração das margens afeta todo documento). 
O caso do bloco (dentro de chaves) é o caso particular aceito
somente para os comandos de seleção de fontes e tamanho de fontes.


\begin{flushright} 
% O ambiente protege mudança de configurações, não somente das fontes.
  O texto está alinhado a direita.
  \center O \verb+\center+ está sendo usando sem o \verb|\begin{}|, \verb|\end{}| 
  por estar dentro do ambiente.
  
  Este e outra configurações serão restauradas no \verb|\end|.

  \flushleft \bfseries \Large negrito grande
\end{flushright}

Devido a este comportamento, as vezes usa o comando de mudança de fontes, colocando dentro do bloco delimitado pelas chaves.
Ex.: {\ttfamily typewriter} dentro do bloco.


{\bfseries Nota para usuário do \LaTeX{} antigo:}
O \LaTeXe \ dispõe do comando de especificação de fontes com duas letras, 
usadas no \LaTeX{} $2.09$ devido a compatibilidade, 
mas eles não podem ser combinados, por ter 
combinação pré-definida: \\
 \texttt{rm} é romano reto de peso normal, \\
 \texttt{tt} é monoespaçado reto de peso normal, \\
 \texttt{sf} é sans serif reto de peso normal, \\
 \texttt{bf} é romano reto negrito, \\
 \texttt{it} é romano itálico de peso normal, \\
 \texttt{sc} é ``small caps'' romano de peso normal, \\
 
Portanto, o uso de comandos de duas letras na especificação das fontes 
devem ser usados somente quando especificações não precisam ser combinados.
Além disso, todos eles são seleção de fontes e não a especificação do trecho.
Assim, deve evitar de usar no meio do texto.
Caso estiver usando por algum motivo, o \verb+\it+ requer 
\verb+\/+ (correção do itálico) no final como em

Um {\it espaço vetorial\/} é o conjunto ...

Se a fonte alterar dentro do itálico para romano, também pode ocurrer
mesmo problema. Assim, a recomendação é sempre usar o \verb+\text??+ e
\verb+\emph+ quando a especificação é efetuada no meio do texto.

\end{document} % final do documento
