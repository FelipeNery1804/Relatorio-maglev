%===============================================================================
% $Id: ifacconf.tex 19 2011-10-27 09:32:13Z jpuente $  
% Template for IFAC meeting papers
% Copyright (c) 2007-2008 International Federation of Automatic Control
%===============================================================================
\documentclass{ifacconf}

\usepackage{graphicx}      % include this line if your document contains figures
\usepackage{natbib}        % required for bibliography
%===============================================================================
\begin{document}
\begin{frontmatter}

\title{Grupo 3 - Módulo 1 - Massa Mola} 
% Title, preferably not more than 10 words.

\thanks[footnoteinfo]{Sponsor and financial support acknowledgment
goes here. Paper titles should be written in uppercase and lowercase
letters, not all uppercase.}

\author[First]{Felipe Nery Barcelos Araújo (2020021190)} 
\author[First]{Gustavo Vieira Barbosa (2020021352)} 
\author[First]{Matheus Marques Gonçalves de Paula (2020068995)}

\address[First]{
  Engenharia de Controle e Automação,\\ Universidade Federal de Minas Gerais, MG \\
   (e-mails: felipenery@ufmg.br, gustavovbarbosa@ufmg.br, mmgp@ufmg.br)
}

%Escrever um resumo do documento aqui, não pode ultrapassar 250 palavras
\begin{abstract}               
These instructions give you guidelines for preparing papers for IFAC
technical meetings. Please use this document as a template to prepare
your manuscript. For submission guidelines, follow instructions on
paper submission system as well as the event website.
\end{abstract}

%Escrever até 5 palavras chave do relatório aqui
\begin{keyword}
Five to ten keywords, preferably chosen from the IFAC keyword list.
\end{keyword}

\end{frontmatter}

%===============================================================================

\section{Introdução}

Os sistemas massa-mola são capazes de representar uma ampla gama de fenômenos
físicos e engenharia. De forma a desempenhar um papel fundamental na modelagem 
desses fenômenos e sistemas mecânicos complexos, tais como: suspensões veiculares, 
sistemas de suspensão de edifícios, sistemas biomecânicos e muitos outros o sistema 
massa-mola representa com elevada exatidão.

\section{Descrição da planta e especificação do desempenho desejado}

Next we see a few subsections.

\subsection{Descrição da planta e especificação do desempenho desejado}

Faça tudo

\section{Modelagem matemática e validação do modelo}

Modeloo

\subsection{Modelagem matemática}

Modelo 2

\subsection{Validação do modelo}

Modelo 3

\section{Projeto do controlador}

Controle credo kkkkkkkkkk

\section{Resultados experimentais em malha fechada}


\section{Conclusão}

Nery vai concluir o trabalho pra gente!

\bibliography{ifacconf}             % bib file to produce the bibliography
                                                     % with bibtex (preferred)
                                                   
%\begin{thebibliography}{xx}  % you can also add the bibliography by hand

%\bibitem[Able(1956)]{Abl:56}
%B.C. Able.
%\newblock Nucleic acid content of microscope.
%\newblock \emph{Nature}, 135:\penalty0 7--9, 1956.

%\bibitem[Able et~al.(1954)Able, Tagg, and Rush]{AbTaRu:54}
%B.C. Able, R.A. Tagg, and M.~Rush.
%\newblock Enzyme-catalyzed cellular transanimations.
%\newblock In A.F. Round, editor, \emph{Advances in Enzymology}, volume~2, pages
%  125--247. Academic Press, New York, 3rd edition, 1954.

%\bibitem[Keohane(1958)]{Keo:58}
%R.~Keohane.
%\newblock \emph{Power and Interdependence: World Politics in Transitions}.
%\newblock Little, Brown \& Co., Boston, 1958.

%\bibitem[Powers(1985)]{Pow:85}
%T.~Powers.
%\newblock Is there a way out?
%\newblock \emph{Harpers}, pages 35--47, June 1985.

%\bibitem[Soukhanov(1992)]{Heritage:92}
%A.~H. Soukhanov, editor.
%\newblock \emph{{The American Heritage. Dictionary of the American Language}}.
%\newblock Houghton Mifflin Company, 1992.

%\end{thebibliography}

\appendix
\section{A summary of Latin grammar}    % Each appendix must have a short title.
\section{Some Latin vocabulary}              % Sections and subsections are supported  
                                                                         % in the appendices.
\end{document}
