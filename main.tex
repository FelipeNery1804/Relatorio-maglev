%===============================================================================
% $Id: ifacconf.tex 19 2011-10-27 09:32:13Z jpuente $  
% Template for IFAC meeting papers
% Copyright (c) 2007-2008 International Federation of Automatic Control
%===============================================================================
\documentclass{ifacconf}

\usepackage{graphicx}      % include this line if your document contains figures
\usepackage{natbib}        % required for bibliography
%===============================================================================
\begin{document}
\begin{frontmatter}

\title{Grupo 3 - Módulo 1 - Massa Mola} 
% Title, preferably not more than 10 words.

%\thanks[footnoteinfo]{Sponsor and financial support acknowledgment
%goes here. Paper titles should be written in uppercase and lowercase
%letters, not all uppercase.}

\author[First]{Felipe Nery Barcelos Araújo (2020021190)} 
\author[First]{Gustavo Vieira Barbosa (2020021352)} 
\author[First]{Matheus Marques Gonçalves de Paula (2020068995)}

\address[First]{
  Engenharia de Controle e Automação,\\ Universidade Federal de Minas Gerais, MG \\
   (e-mails: felipenery@ufmg.br, gustavovbarbosa@ufmg.br, mmgp@ufmg.br)
}

%Escrever um resumo do documento aqui, não pode ultrapassar 250 palavras
\begin{abstract}               
These instructions give you guidelines for preparing papers for IFAC
technical meetings. Please use this document as a template to prepare
your manuscript. For submission guidelines, follow instructions on
paper submission system as well as the event website.
\end{abstract}

%Escrever até 5 palavras chave do relatório aqui
\begin{keyword}
Five to ten keywords, preferably chosen from the IFAC keyword list.
\end{keyword}

\end{frontmatter}

%===============================================================================

\section{Introdução}

Os sistemas massa-mola são capazes de representar uma ampla gama de fenômenos
físicos e engenharia. De forma a desempenhar um papel fundamental na modelagem 
desses fenômenos e sistemas mecânicos complexos, tais como: suspensões veiculares, 
sistemas de suspensão de edifícios, sistemas biomecânicos e muitos outros o sistema 
massa-mola representa com elevada exatidão. Afinal, com uma modelagem precisa, o sistema
é capaz de realizar controle garantindo o desempenho e a estabilidade desejada.

Ao longo desse relatório será visto um estudo focado no controle de um sistema de massa-mola 
composto por duas massas interconectadas por duas molas, como mostra a figura \ref{fig:planta_padrao}, um problema
clássico de controle de sistemas dinâmicos. 

\begin{figure}[!htb]
  \begin{center}
  \includegraphics[width=8.4cm]{figures/planta_padrao.jpg}    % The printed column width is 8.4 cm.
  \caption{Figura da planta real estudada, composta por duas molas de diferentes coeficientes e massas de 500 gramas cada. Fonte: Autoral} 
  \label{fig:planta_padrao}
  \end{center}
\end{figure}

Com isso, nas seções subsequentes, exploraremos em detalhes a modelagem matemática do sistema
massa-mola, bem como o fundamento do controle utilizado e demonstrações práticas.

\section{Descrição da planta e especificação do desempenho desejado}

Next we see a few subsections.

\subsection{Descrição da planta e especificação do desempenho desejado}

Faça tudo

\section{Modelagem matemática e validação do modelo}

Modeloo

\subsection{Modelagem matemática}

Modelo 2

\subsection{Validação do modelo}

Modelo 3

\section{Projeto do controlador}

A principio, buscamos realizar o controlador proporcional, integrativo e derivativo (PID), 
por garantir erro nulo para entrada em degrau e por ser amplamente difundido nas industrias
e sistemas de controle em geral. A seguir será explicitado as tentativas para obter esse controlador.

\subsection{Primeira tentativa} %Falha obtida
Para obtenção dos parâmetros foi plotado o lugar das raizes da planta em malha fechada, figura X.
Com o lugar da raizes traçados, foi utilizado o \textit{sisotool} 

\subsection{Segunda tentiva} %Controle corrigido
Sequencialmente após a correção dos parâmetros C1 e C2, de forma impirica, obtendo as seguintes
matrizes do espaço de estados:
Posteriormente foi realizado um novo estudo sobre o lugar das raizes em malha fechada, figura \ref{fig:lugar_raizes_mf_atualizada}, em que
aproximando para próximo da origem temos a figura \ref{fig:lugar_raizes_mf_atualizada_zoom}. 
Em que é possível perceber que a caracteristica da resposta do modelo é fortemente marcada por dois pares de polos conjugados
próximos a origem levando para a instabilidade do sistema, como mostra a resposta ao degrau do sistema, figura \ref{fig:resposta_Degrau_mf_atualizada}.

\begin{figure}[!htb]
  \begin{center}
  \includegraphics[width=8.4cm]{figures/lugar_raizes_mf_corrigida.png}    % The printed column width is 8.4 cm.
  \caption{Lugar da raizes da malha fechada da planta atualizada. Fonte: autoral, produzida com \textit{matlab}.} 
  \label{fig:lugar_raizes_mf_atualizada}
  \end{center}
\end{figure}

\begin{figure}[!htb]
  \begin{center}
  \includegraphics[width=8.4cm]{figures/lugar_raizes_mf_corrigida_aproximada.png}    % The printed column width is 8.4 cm.
  \caption{Lugar da raizes da malha fechada da planta atualizada aproximada na origem. Fonte: autoral, produzida com \textit{matlab}.} 
  \label{fig:lugar_raizes_mf_atualizada_zoom}
  \end{center}
\end{figure}

\begin{figure}[!htb]
  \begin{center}
  \includegraphics[width=8.4cm]{figures/resposta_degrau_mf_atualizado.png}    % The printed column width is 8.4 cm.
  \caption{Resposta ao degrau da planta atualizada. Fonte: autoral, produzida com \textit{matlab}.} 
  \label{fig:resposta_Degrau_mf_atualizada}
  \end{center}
\end{figure}

%Falar agora do controle via lugar das raizes, foco em não ter sobressinal e estabilização em menos de 4s e falar dos parâmetro obtidos

\section{Resultados experimentais em malha fechada}


\section{Conclusão}

Nery vai concluir o trabalho pra gente!

\bibliography{ifacconf}             % bib file to produce the bibliography
                                                     % with bibtex (preferred)
                                                   
%\begin{thebibliography}{xx}  % you can also add the bibliography by hand

%\bibitem[Able(1956)]{Abl:56}
%B.C. Able.
%\newblock Nucleic acid content of microscope.
%\newblock \emph{Nature}, 135:\penalty0 7--9, 1956.

%\bibitem[Able et~al.(1954)Able, Tagg, and Rush]{AbTaRu:54}
%B.C. Able, R.A. Tagg, and M.~Rush.
%\newblock Enzyme-catalyzed cellular transanimations.
%\newblock In A.F. Round, editor, \emph{Advances in Enzymology}, volume~2, pages
%  125--247. Academic Press, New York, 3rd edition, 1954.

%\bibitem[Keohane(1958)]{Keo:58}
%R.~Keohane.
%\newblock \emph{Power and Interdependence: World Politics in Transitions}.
%\newblock Little, Brown \& Co., Boston, 1958.

%\bibitem[Powers(1985)]{Pow:85}
%T.~Powers.
%\newblock Is there a way out?
%\newblock \emph{Harpers}, pages 35--47, June 1985.

%\bibitem[Soukhanov(1992)]{Heritage:92}
%A.~H. Soukhanov, editor.
%\newblock \emph{{The American Heritage. Dictionary of the American Language}}.
%\newblock Houghton Mifflin Company, 1992.

%\end{thebibliography}

\appendix
\section{A summary of Latin grammar}    % Each appendix must have a short title.
\section{Some Latin vocabulary}              % Sections and subsections are supported  
                                                                         % in the appendices.
\end{document}
